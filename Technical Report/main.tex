\documentclass[14pt]{article}
\usepackage{listings}
\usepackage{color}
\usepackage{graphicx}
\usepackage{setspace}

\definecolor{dkgreen}{rgb}{0,0.6,0}
\definecolor{gray}{rgb}{0.5,0.5,0.5}
\definecolor{mauve}{rgb}{0.58,0,0.82}

\lstset{frame=tb,
  language=C,
  aboveskip=3mm,
  belowskip=3mm,
  showstringspaces=false,
  columns=flexible,
  basicstyle={\small\ttfamily},
  numbers=none,
  numberstyle=\tiny\color{gray},
  keywordstyle=\color{blue},
  commentstyle=\color{dkgreen},
  stringstyle=\color{mauve},
  breaklines=true,
  breakatwhitespace=true
  tabsize=3
}







\begin{document}
\title{\huge Lego Game (Assigment 14)}
\maketitle
\begin{center}
\vspace{30 mm}

\title{\huge Student: Ghibuci Andrei}
\\\vspace{10 mm}
\title{\huge Teacher: Badica Costin}
\\\vspace{10 mm}
\title{\huge Assistant Teacher: Becheru Alexandru}
\\\vspace{10 mm}
\title{\huge Computer science in English}
\\\vspace{10 mm}
\title{\huge Group 1.2 A}
\\\vspace{10 mm}
\title{\huge 1st Year}
\maketitle

\newpage
\section*{Introduction}
\vspace{20 mm}
A depth-first search (DFS) is an algorithm for traversing a finite graph. DFS visits the child nodes before visiting the sibling nodes; that is, it traverses the depth of any particular path before exploring its breadth. A stack (often the program's call stack via recursion) is generally used when implementing the algorithm.
\\\vspace{10 mm}
Topological sorting for Directed Acyclic Graph (DAG) is a linear ordering of vertices such that for every directed edge uv, vertex u comes before v in the ordering. Topological Sorting for a graph is not possible if the graph is not a DAG.
For example, a topological sorting of the following graph is “5 4 2 3 1 0”. There can be more than one topological sorting for a graph. For example, another topological sorting of the following graph is “4 5 2 3 1 0”. The first vertex in topological sorting is always a vertex with in-degree as 0 (a vertex with no incoming edges).

\newpage
\end{center}
\section*{Problem statement}
Alexander, which is a toddler, has received as a birthday gift a Lego game. How-
ever, he was not able to construct the toy Lego, as there are many pieces. Alexan-
der has observed that each piece is numbered with a number starting from 1 to
1000. The instructions of the game state the pieces that need to be assembled
before each an every piece. Help Alexander to build his Lego toy by developing
an application which determines the correct order in which the piece of Lego
need to be assembled. he application should implement two different algorithms
to help Alexander.

So, I had to make an orientated acyclic graph and tropological sort it. However, I wasn't able to make it sort from 1 to 1000, so i adapted it to sort a part of them. The problem was that I couldn't fix two bugs. First bug was when my graph was generated even( source node == destination node) and the second one was when my graph returned to the begining node too soon( source[1] - destination[2], destination[2] - source[1]).

\newpage
\section*{Pseudocode}
First, we will use an adjacency matrix for the directed graph representation, being an easier method. And another two adjaceny vectors for the source and destination nodes.
\\
We will use functions for each operation required by the application, with integer parameters, which will be called by the main program.
\\
Here are the important functions employed by the program:
\begin{lstlisting}
orientations (int oreintation[][], int i, int n)
1.	int source
2.	int destination
3.	srand(time(NULL))
4.	while i <- 1 to n execute
	    4.1 source = random generation between 1 and 1000
	    4.2 destination = random generation between 1 and 1000
	    4.3 while adjancency_source[source] = 1 execute
	            4.3.1 source = random generation between 1 and 1000
	    4.4 ajancency_source[source] = 1   
	    4.5 while adjancency_destination[destination] = 1 execute
	            4.5.1 destination = random generation between 1 and 1000
	    4.6 adjancency_destination[destination] = 1
	    4.7 orientation[source][destination] = 1
	    4.8 i++
	    
\end{lstlisting}
\begin{lstlisting}
depth_first_search(int source)
1.	int destination
2.	int i
3.	reach[source] = 1
4.	for i <- 1 to 1000 do
       4.1 if orientation[source][i] = 1 and reach[i] = 0 then
                4.1.1 output source <- i
                4.1.2 counter++
                4.1.3 depth_first_search(i)
\end{lstlisting}

\newpage
\section*{Application design}
\vspace{10 mm}
The library contains the header \textbf{\textit{functions.h}} which has all the function prototypes to compute the required operations. These are all of them:
\\---void orientations(int orientation[][], int i , int n)
\\---void depth first search(int node)
\\\vspace{3 mm}
\\The source file \textbf{\textit{functions.c}} has all the function implementations.
\\Function depth first search(int node) includes some steps, like:
\\1) We visit the starting node and print it,\\
2) We visit the nearest unvisited neighbor that can be reached and do the same for that one,
\\3) If we exhaust all our options for a node, we return back to the previous one, repeating step \textbf{\textit{2}}.
\\\vspace{3 mm}
\\ The adjacency matrix is stored as a 2D array called 'orientation'. This array is used by all of our functions. We will use a "source" node and a "destination" one, with the value '0' for unconected graph and 1 for orientation from source to destination. \textbf{\textit{source}} represents the starting node from where we want to begin the traversal of the graph. \textbf{\textit{n}} is simply the number of nodes. The function works recursively.
\\\vspace{2 mm}
\\Function Orientations(int orientation[][1001], int i, int n) - works as follows: 
\\1) We will random generate the source and the destination node,\\
2) Next we will check the adjancecy matrices of source and destination nodes to see if the random generation was allready generated ,
\\3) After we checked and assigned the random nodes we assign 1 to the orientation matrix which means the conection from the source node to destination node is available and checked .

\vspace{3 mm}
The final source file, \textbf{\textit{main.c}} itself, uses all of the functions from the \textbf{\textit{functions.c}}.

After calling all the functions, the main function will randomly generate a tropological sort between 1 and 1000 which represents the correct order of the Lego. As for the bugs as mentioned I tried to fix them and wasn't able to.
\newpage
\section*{Source Code}
\begin{lstlisting}
//-----------------------------functions.h-------------------------------

ifndef FUNCTIONS_H_INCLUDED
#define FUNCTIONS_H_INCLUDED

void orientations ( int orientation[][1001] , int i , int n);
void depth_first_search( int node );

#endif // FUNCTIONS_H_INCLUDED

\end{lstlisting}
\begin{lstlisting}
//-----------------------------functions.c-------------------------------
#include "functions.h"
#include <stdio.h>
#include <stdlib.h>
#include <time.h>

/**
* This vector is a representation of the current node and if this node was allready visited.
*/
int reach[1001];

/**
* This variable will be used to counter how many pieces were used in the sort.
*/
int counter = 0;

/**
* A matrix that retain the orientations between the nodes of the graph.
*/
int orientation[1001][1001];

/**
* Vector that retain if the source node allready appeared in the sort
*/
int adiacency_source[1001];

/**
* Vector that retain if the destination node allready appeared in the sort
*/
int adjacency_destination[1001];

/**
* Function that reads and makes the graph orientated.
*/
void orientations( int orientation [][1001] , int i , int n ){

    int source;
    int destination;

    srand(time(NULL));

    while( i <= n){

         source = rand()%1000 + 1;
         destination = rand()%1000 + 1;

         while( adjacency_source[source] == 1 ) //Randomly generate source node if it haven't allready occured.
               source = rand()%1000 + 1;
        adjacency_source[source] = 1;

         while(adjacency_destination[destination] == 1 )// Randomly generate destination node if it haven't allready occured.
               destination = rand()%1000 + 1;
         adjacency_destination[destination] = 1;

         orientation[source][destination] = 1; // Assigning the orietation from the source to node.
         i++;
        }
}

/**
* This function will do the tropological sort of the graph.
*/
void depth_first_search(int source){

     int destination;
     int i;
     reach[source]=1;

     for(i = 1;i <= 1000; i++)
         if(orientation[source][i] && !reach[i]){

            printf("\n %d->%d",source,i);// Printing the tropological sort.
            counter++;// Counting the number of pieces used.
            depth_first_search(i);
            }
}

//---------------------------------main.c--------------------------------


#include "functions.h"
#include <stdio.h>
#include <stdlib.h>

/**
* Matrix that was used allready in the function orientations and explained.
*/
int orientation[1001][1001];

/**
*Variable that was used for counting the number of used pieces.
*/
int counter;

/**
* Main function of my project
*/
int main()
{
    int source;
    int destination;

    orientations(orientation, 1, 1000);
    depth_first_search(rand()%1000+1);

    printf("\n %d ", counter);

    return 0;
}


\end{lstlisting}

\newpage
\section*{Experiments and results}
\begin{lstlisting}
 First Experiment: 
 1000 nodes
 Starting node: 511
 Pieces Used : 676 
 Tropological Sort :
 511->506
 506->622
 622->986
 986->562
 562->729
 729->976
 976->685
 685->678
 678->424
 424->353
 353->546
 546->411
 411->75
 75->571
 571->398
 398->612
 612->788
 788->437
 437->130
 130->963
 963->91
 91->115
 115->818
 818->826
 826->136
 136->807
 807->478
 478->652
 652->857
 857->973
 973->553
 553->701
 701->234
 234->863
 863->756
 756->387
 387->53
 53->854
 854->423
 423->727
 727->493
 493->899
 899->885
 885->327
 327->737
 737->811
 811->391
 391->11
 11->274
 274->714
 714->375
 375->558
 558->590
 590->797
 797->12
 12->4
 4->734
 734->214
 214->721
 721->650
 650->665
 665->560
 560->994
 994->175
 175->830
 830->408
 408->455
 455->909
 909->592
 592->331
 331->314
 314->535
 535->828
 828->654
 654->853
 853->469
 469->25
 25->956
 956->138
 138->141
 141->207
 207->122
 122->939
 939->877
 877->472
 472->522
 522->577
 577->799
 799->787
 787->540
 540->95
 95->180
 180->252
 252->279
 279->266
 266->709
 709->931
 931->428
 428->172
 172->705
 705->232
 232->895
 895->216
 216->468
 468->741
 741->657
 657->176
 176->57
 57->247
 247->218
 218->444
 444->954
 954->121
 121->431
 431->246
 246->90
 90->611
 611->941
 941->840
 840->168
 168->498
 498->969
 969->583
 583->171
 171->948
 948->570
 570->450
 450->621
 621->503
 503->712
 712->849
 849->846
 846->587
 587->98
 98->667
 667->38
 38->415
 415->873
 873->51
 51->964
 964->396
 396->317
 317->333
 333->183
 183->674
 674->959
 959->618
 618->221
 221->212
 212->832
 832->866
 866->322
 322->981
 981->740
 740->273
 273->62
 62->458
 458->166
 166->211
 211->526
 526->77
 77->362
 362->257
 257->524
 524->915
 915->512
 512->46
 46->692
 692->835
 835->329
 329->951
 951->888
 888->361
 361->482
 482->624
 624->109
 109->600
 600->249
 249->297
 297->975
 975->842
 842->970
 970->920
 920->263
 263->215
 215->120
 120->949
 949->356
 356->148
 148->702
 702->579
 579->201
 201->708
 708->5
 5->716
 716->655
 655->301
 301->283
 283->81
 81->377
 377->3
 3->922
 922->984
 984->957
 957->160
 160->443
 443->449
 449->489
 489->719
 719->344
 344->633
 633->118
 118->623
 623->94
 94->289
 289->784
 784->372
 372->608
 608->287
 287->241
 241->802
 802->230
 230->378
 378->881
 881->494
 494->615
 615->549
 549->161
 161->298
 298->271
 271->812
 812->629
 629->995
 995->52
 52->547
 547->192
 192->661
 661->892
 892->193
 193->574
 574->809
 809->154
 154->23
 23->958
 958->591
 591->236
 236->292
 292->367
 367->731
 731->328
 328->613
 613->658
 658->36
 36->924
 924->41
 41->258
 258->580
 580->805
 805->593
 593->61
 61->785
 785->54
 54->531
 531->996
 996->806
 806->564
 564->823
 823->47
 47->770
 770->744
 744->324
 324->859
 859->345
 345->649
 649->780
 780->673
 673->746
 746->280
 280->101
 101->476
 476->500
 500->227
 227->107
 107->240
 240->764
 764->332
 332->217
 217->244
 244->461
 461->371
 371->906
 906->79
 79->149
 149->132
 132->194
 194->738
 738->277
 277->89
 89->599
 599->833
 833->142
 142->750
 750->769
 769->484
 484->433
 433->186
 186->988
 988->491
 491->831
 831->554
 554->879
 879->838
 838->42
 42->474
 474->262
 262->893
 893->700
 700->720
 720->688
 688->874
 874->203
 203->27
 27->627
 627->771
 771->382
 382->92
 92->572
 572->852
 852->543
 543->946
 946->926
 926->773
 773->697
 697->518
 518->792
 792->291
 291->872
 872->261
 261->516
 516->426
 426->908
 908->856
 856->992
 992->960
 960->901
 901->589
 589->275
 275->735
 735->595
 595->97
 97->164
 164->358
 358->439
 439->33
 33->602
 602->363
 363->430
 430->816
 816->878
 878->687
 687->417
 417->887
 887->584
 584->766
 766->453
 453->173
 173->418
 418->153
 153->707
 707->238
 238->169
 169->349
 349->537
 537->393
 393->604
 604->783
 783->753
 753->309
 309->517
 517->789
 789->432
 432->127
 127->556
 556->841
 841->684
 684->913
 913->929
 929->189
 189->670
 670->58
 58->225
 225->30
 30->388
 388->870
 870->985
 985->1
 1->847
 847->413
 413->436
 436->989
 989->145
 145->475
 475->732
 732->462
 462->233
 233->268
 268->29
 29->585
 585->337
 337->134
 134->364
 364->253
 253->594
 594->359
 359->569
 569->798
 798->155
 155->206
 206->219
 219->559
 559->405
 405->198
 198->717
 717->796
 796->616
 616->967
 967->267
 267->900
 900->794
 794->777
 777->282
 282->943
 943->269
 269->601
 601->982
 982->573
 573->542
 542->447
 447->406
 406->695
 695->610
 610->754
 754->187
 187->286
 286->752
 752->538
 538->736
 736->536
 536->284
 284->485
 485->745
 745->776
 776->588
 588->614
 614->243
 243->321
 321->26
 26->747
 747->373
 373->952
 952->477
 477->758
 758->545
 545->603
 603->32
 32->821
 821->104
 104->210
 210->576
 576->341
 341->945
 945->181
 181->307
 307->167
 167->124
 124->69
 69->507
 507->72
 72->795
 795->979
 979->827
 827->530
 530->380
 380->669
 669->177
 177->728
 728->311
 311->401
 401->814
 814->916
 916->384
 384->195
 195->897
 897->942
 942->20
 20->551
 551->715
 715->761
 761->755
 755->668
 668->404
 404->190
 190->864
 864->204
 204->786
 786->868
 868->93
 93->682
 682->446
 446->222
 222->567
 567->822
 822->44
 44->965
 965->144
 144->646
 646->235
 235->824
 824->191
 191->810
 810->643
 643->778
 778->937
 937->568
 568->129
 129->100
 100->343
 343->205
 205->70
 70->239
 239->354
 354->947
 947->386
 386->921
 921->8
 8->351
 351->442
 442->159
 159->165
 165->596
 596->73
 73->950
 950->74
 74->925
 925->582
 582->394
 394->131
 131->793
 793->495
 495->200
 200->914
 914->889
 889->641
 641->40
 40->689
 689->660
 660->938
 938->112
 112->676
 676->119
 119->883
 883->653
 653->76
 76->907
 907->202
 202->762
 762->515
 515->66
 66->923
 923->250
 250->790
 790->791
 791->837
 837->102
 102->765
 765->726
 726->933
 933->927
 927->834
 834->113
 113->628
 628->869
 869->85
 85->487
 487->607
 607->370
 370->894
 894->414
 414->803
 803->128
 128->999
 999->340
 340->690
 690->178
 178->464
 464->260
 260->904
 904->24
 24->813
 813->158
 158->15
 15->903
 903->110
 110->397
 397->918
 918->686
 686->898
 898->910
 910->199
 199->638
 638->742
 742->185
 185->326
 326->152
 152->318
 318->126
 126->725
 725->779
 779->248
 248->224
 224->366
 366->285
 285->251
 251->303
 303->105
 105->617
 617->302
 302->265
 265->163
 163->223
 223->365
 365->402
 402->473
 473->497
 497->264
 264->631
 631->640
 640->441
 441->335
 335->421
 421->972
 972->281
 281->488
 488->699
 699->730
 730->480
 480->325
 325->336
 336->886
 886->63
 63->147
 147->706
 706->319
 319->316
 316->862
 862->998
 998->278
 278->775
 775->917
 917->672
 672->348
 348->323
 323->713
 713->642
 642->486
 486->991
 991->103
 103->808
 808->858
 858->748
 748->288
 288->87
 87->125
 125->179
 179->146
 146->465

2. 10 nodes
The tropological sort is :
 7->2
 2->4
 4->5
 5->8
 The number of pieces used : 4
\end{lstlisting}


\newpage
\section*{Conclusions}
\vspace{20 mm}
Personally, It was a challenging assigment for my programming skills. I tried hard to complete the tasks and I was a little dissapointed that I couldn't make the program work as it should have. However, I developed a software which I am proud of, I worked hard and I tried my best and I think these kind of tasks are a great choice to assign to students. It was a pleasure to take part in such assigment. 
\\\vspace{20mm}
\section*{References}
\large 1)https://www.geeksforgeeks.org/topological-sorting/
\\
\\\vspace{6mm}
\\
2)https://en.wikipedia.org/wiki/Depth-first-search
\\
\\\vspace{6mm}
\\
3)http://www.sharelatex.com


\end{document}
